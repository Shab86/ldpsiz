\# \-Introduction \-Ldpsiz is a computer package that estimates the history of population size from data on linkage disequilibrium (\-L\-D).

\-L\-D is estimated in terms of the \char`\"{}standard linkage deviation\char`\"{}, $\sigma_d^2$ (\-Ohta and \-Kimura, 1971, \-Genetics 68\-: 571-\/580). \-This parameter is usually interpreted as an approximation to the expected value of $r^2$, the squared correlation between loci. $\sigma_d^2$ is a weighted average of $r^2$ values, where the weight given each pair of loci is the product of the two heterozygosities. \-Because of these weights, $\sigma_d^2$ is not affected much by loci near fixation. \-For this reason, it is ideal for use with low-\/coverage sequence data. \-Most of the sequencing errors in such data show up as singleton loci, which have little effect on $\sigma_d^2$.

\-To estimate population history parameters, `ldpsiz` minimizes the difference between values of $\sigma_d^2$ estimated from data and values predicted from population history. \-The predicted values are obtained from the recurrence equations either of \-Hill (1975, \-Theoretical \-Population \-Biology 8\-:117-\/126) or of \-Strobeck and \-Morgan (1978, \-Genetics 88\-:829-\/844). \-This calculation can be done in two different ways. \-The slow way (invoked using the `-\/-\/exact` option of \hyperlink{preld}{preld}) iterates the difference equations one generation at a time. \-The default method speeds things up by solving differential equations that approximate the difference equations. \-These methods are fast enough to allow one to examine a large number of hypotheses about history parameters.

\-To model population history, `ldpsiz` divides the past into a fixed number of epochs. \-Within each epoch, the population is assumed constant in size. \-There is no provision for multiple populations in the current version of the code. \-In the earliest epoch, the population is assumed to be at equilibrium with respect to genetic drift, mutation, and recombination.

\-Uncertainties are estimated using a moving-\/blocks bootstrap. \-In calculating $\sigma_d^2$, most operations are done only once for each pair of sites and are not repeated for each bootstrap replicate. \-Furthermore, calculations are parallelized. \-These provisions make it possible to use hundreds of bootstrap replicates in genome-\/scale analyses, on computers with large numbers of \-C\-P\-U cores.

\# \-Installation and testing

\-The package is available at \mbox{[}github\mbox{]}(github.\-com/alanrogers/ldpsiz) in various formats. \-Before compiling, you must install two libraries\-: `pthreads` and \mbox{[}`gsl`\mbox{]}(\href{http://www.gnu.org/software/gsl}{\tt http\-://www.\-gnu.\-org/software/gsl}). \-You will need not only the libraries themselves but also several header files, such as `pthread.h`. \-I didn't need to install `pthreads`, because it came bundled with the \-Gnu \-C compiler. \-But the gsl was an extra. \-Under ubuntu \-Linux, you can install it like this\-:

sudo apt-\/get install libgsl0-\/dev

\-On the mac, using homebrew, the command is

brew install gsl

\-By default, the executable files will be copied into a directory named `bin` in your home directory. \-If you want them to go somewhere else, edit the first non-\/comment line of src/\-Makefile.

\-Then

1. \-Cd into the src directory. 2. \-Type \char`\"{}make\char`\"{}. 3. \-Type \char`\"{}make install\char`\"{}.

\-This will try to place the executables into directory \char`\"{}bin\char`\"{} in the user's home directory. \-Make sure this directory appears in your \-P\-A\-T\-H, so that the shell can find it.

\-This installation will work under unix-\/like operating systems, such as linux and \-Apple's osx. \-I haven't tried to port this software to \-Windows.

\-The directory `test` contains a unit test for many of the .c files in directory `src`. \-Within this directory, type

1. make xhill 2. ./xhill

to test the source file `hill.c`. \-To run all unit tests, type \char`\"{}make\char`\"{}. \-This will take awhile, as some of the unit tests are slow.

\-The unit tests will not compile if `\-N\-D\-E\-B\-U\-G` is defined. \-If optimization is turned on during optimization, some of the unit tests may be removed by the optimizer. \-To avoid these problems, comment out the relevant line at the top of src/\-Makefile.

\# \-Genetic input data

`ldpsiz` reads data files in \char`\"{}gtp\char`\"{} format. \-In the distribution, the names of all such files end with the suffix \char`\"{}.\-gtp\char`\"{}. \-Here are the first few lines of a sample .gtp file containing haploid data\-:

\# data source = /\-Users/rogers/bin/macs2gtp.py out \# time = 2012-\/05-\/28 09\-:28\-:11.\-098762 \# ploidy = 1 \# snp\-\_\-id nucpos mappos alleles genotypes 0 262 0.\-0262 \-A\-G 0001000000 1 362 0.\-0362 \-A\-T 0000010000 2 536 0.\-0536 \-T\-C 1000000000 3 799 0.\-0799 \-T\-A 0010000000 4 861 0.\-0861 \-C\-T 0000000100

\-The first 3 lines are optional, but the line of column headers is not. \-After the header (the lines beginning with '\#'), each line refers to a single-\/nucleotide polymorphism (a \hyperlink{struct_s_n_p}{\-S\-N\-P}). \-The columns are as follows\-:

1. snp\-\_\-id is an arbitrary label associated with the \hyperlink{struct_s_n_p}{\-S\-N\-P}. \-These labels are not used. 2. nucpos \-The position of the \hyperlink{struct_s_n_p}{\-S\-N\-P} on the chromosome, measured in nucleotides. 3. mappos \-The position of the \hyperlink{struct_s_n_p}{\-S\-N\-P} on the chromosome, measured in centimorgans (c\-M). 4. \-The alleles present at this \hyperlink{struct_s_n_p}{\-S\-N\-P} locus (for example, \-A\-C or 01). 5. \-The state of each individual in the sample. \-State is always either 0 or 1. \-A \char`\"{}1\char`\"{} indicates that the allele at this position was the first allele in the list of alleles. \-A \char`\"{}0\char`\"{} indicates any of the alternate alleles. \-For example, if the alleles list was \char`\"{}\-A\-T\char`\"{}, then genotypes \char`\"{}0110\char`\"{} mean \char`\"{}\-T\-A\-A\-T\char`\"{}. \-There is no provision for missing values.

\-For diploid data, the header must contain the line

\# ploidy = 2

\-For diploid data, each phased genotype is a $\ast$pair$\ast$ of adjacent character values. \-Homozygotes look like \char`\"{}00\char`\"{} or \char`\"{}11\char`\"{}, and phased heterozygotes look like \char`\"{}01\char`\"{} or \char`\"{}10\char`\"{}. \-Unphased heterozygotes look like \char`\"{}h\char`\"{} and occupy just a single character position.

\# \-Initialization file

\-Many of the programs read an initialization file called `ldpsiz.ini`, which can be used to set various parameters and to specify a population history. \-This makes it easy to use a single set of parameters in a variety of operations, involving different programs within the package.

\-Here is an example initialization file\-:

\# \-Comments go from \char`\"{}\#\char`\"{} to end of line. blocksize = 300 \# \-S\-N\-Ps per block in bootstrap bootfilename = eld.\-boot bootreps = 25 \# number of bootstrap replicates recombination = 2e-\/7 \# per site per generation mutation = 1e-\/7 \# per site per generation confidence = 0.\-8 \# size of confidence interval window\-Cm = 0.\-1 \# maximum difference (c\-M) btw \-S\-N\-Ps compared lo\-Cm = 0.\-0 \# lowest recombination rate in centimorgans hi\-Cm = 0.\-3 \# highest methods = \-Hill \# models of evolution; comma-\/separated nbins = 25 \# number of bins, each referring to a \# narrow range of recombination rates nthreads = 3 \# number of threads two\-Nsmp = 30 \# number of gene copies in sample

\# \-Population history must come last in the file. \-In begins \# with a line containing the single word \char`\"{}\-Pop\-Hist\char`\"{}. \# \-Each row defines an epoch of population history. \-The first row \# refers to the most recent epoch. \# \-Column 1\-: length of epoch in generations \# \-Column 2\-: haploid population size, 2\-N \hyperlink{struct_pop_hist}{\-Pop\-Hist} \# generations two\-N 40 1e5 \# epoch 0 \-Inf 1e2 \# epoch 1

\-Parameters specified in `ldpsiz.ini` can be overridden by command-\/line arguments. \-Default values are provided for parameters not specified in either of these ways.

\# \-Usage

\-The `ldpsiz` package is not a single monolithic program, but rather a suite of programs. \-These programs are designed to be run at the command line\-: there is no graphical user interface. \-Each program will print a usage message in response to the command-\/line argument `-\/-\/help` or `-\/h`. \-The various programs are described briefly below. \-For more detail, see the documentation for each individual program.

\#\# \-Converting genetic data into .gtp format

\-The package provides several programs for converting into .gtp format. `ms2gtp` and `macs2gtp.py` convert simulated data produced by `ms` and `macs`, and `vcf2gtp.py` converts the vcf-\/format data provided by the 1000 \-Genomes \-Project. \-These programs are written in \-Python, except for `ms2gtp`, which is in \-C.

\#\#\# \-Converting data produced by `ms`

`ms` is \-Dick \-Hudson's program for doing coalescent simulations with recombination. \-It can be downloaded from \-Dick's \mbox{[}homepage\mbox{]}(\href{http://home.uchicago.edu/rhudson1/source.html}{\tt http\-://home.\-uchicago.\-edu/rhudson1/source.\-html}). \-The `ldpsiz` package provides a program, \hyperlink{ms2gtp}{ms2gtp}, which converts the output of `ms` into .gtp format.

\-This program is written in \-C, because it involves transposing a large matrix. \-The original version, written in \-Python, was slow.

\#\#\# \-Converting data produced by `macs`

\-Gary \-Chen's program, `macs`, uses an approximation to the coalescent algorithm, which makes it possible to simulate genome-\/scale data very fast. \-You can get it from \-Gary's \mbox{[}homepage\mbox{]}(\href{http://www-hsc.usc.edu/~garykche}{\tt http\-://www-\/hsc.\-usc.\-edu/$\sim$garykche}). \-To translate `macs` output into .gtp format, use the program \hyperlink{macs2gtp}{macs2gtp}.

\#\#\# \-Converting data from the 1000-\/\-Genomes \-Project

\-To convert data files from the 1000-\/\-Genomes \-Project, use the program \hyperlink{vcf2gtp}{vcf2gtp}.

\#\# \-Predicting $\sigma_d^2$ from population history

\-See the documentation for \hyperlink{preld}{preld}.

\#\# \-Estimating $\sigma_d^2$

\-See the documentation for \hyperlink{eld}{eld}.

\#\# \-Estimating parameters of population history

\-See the documentation for \hyperlink{sald}{sald}.

\#\# \-Examining the cost function.

\-See the documentation for \hyperlink{mcmcld}{mcmcld}. 