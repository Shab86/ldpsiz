Convert .vcf files into .gtp format.

\subsection*{\hyperlink{vcf2gtp_8py}{vcf2gtp.\+py}, a program that converts .vcf files into .gtp format }

\begin{DoxyVerb}usage: vcf2gtp [options filename.vcf geneticMapFile]
where options may include:

  -r <x> or --recombPerBP <x>     Set recombination rate per nucleotide.

If two filenames are provided, the first must be a vcf file and the
second a file of recombination rates. In this case, the -r option is
not allowed.

If one filename is provided, it must be a vcf file, and the -r option
is mandatory.

If no filename is provided, the program reads vcf data from standard
input and the -r option is mandatory.

The program writes to standard output.
\end{DoxyVerb}


If no file is provided for recombination rates, the rate must be specified using the {\ttfamily -\/r} or {\ttfamily -\/-\/recomb\+Per\+B\+P} option, and the program generates map distances assuming that the recombination rate between two sites is proportional to the number of nucleotides that separate them.

The genetic map file must have a format like this\+: \begin{DoxyVerb}position COMBINED_rate(cM/Mb) Genetic_Map(cM)
14431347 9.6640973708 0
14432618 9.7078062447 0.0122830678
14433624 9.7138922111 0.0220491208
14433659 9.7163435060 0.0223891071
14433758 9.7078087850 0.0233510251
\end{DoxyVerb}


The first line is a list of column labels. After that, each row corresponds to a \hyperlink{struct_s_n_p}{S\+N\+P}. The first column gives the position, in nucleotides from the end of the chromosome. The second column gives the recombination rate, in c\+M per Mb, between that \hyperlink{struct_s_n_p}{S\+N\+P} and the next one, and the third column gives the map distance in c\+M. This is the format used by the 1000-\/\+Genomes Project

\begin{DoxyAuthor}{Author}
Ryan Bohlender and Alan Rogers 
\end{DoxyAuthor}
\begin{DoxyCopyright}{Copyright}
Copyright (c) 2014, Ryan Bohlender and Alan R. Rogers \href{mailto:rogers@anthro.utah.edu}{\tt rogers@anthro.\+utah.\+edu}. This file is released under the Internet Systems Consortium License, which can be found in file \char`\"{}\+L\+I\+C\+E\+N\+S\+E\char`\"{}. 
\end{DoxyCopyright}
