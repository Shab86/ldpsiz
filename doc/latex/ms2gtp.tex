This file contains code related ms2gtp, a program that translates output from Dick Hudson's {\ttfamily ms} program into .gtp format.

\subsection*{ms2gtp, a program that converts {\ttfamily ms} output into .gtp format }

This program is written in C, because the conversion involves transposing a large matrix, and this is faster in C. If you run it like \char`\"{}ms2gtp my\+\_\+input\+\_\+file\+\_\+name\char`\"{}, it will allocate internal arrays using default values. If your {\ttfamily ms} output is large, these arrays will be too small. The program will print an error message and abort. To avoid this problem, run the (unix or linux) command \char`\"{}grep positions
ms.\+out $\vert$ wc\char`\"{}. This will print 3 numbers. The number of tokens must be as large as 2nd; buffer size as large as the 3rd. You can set these values using the {\ttfamily -\/-\/max\+Tokens} and {\ttfamily -\/-\/in\+Buff\+Size} arguments.

For example, I used {\ttfamily ms} to produce a file called {\ttfamily ms.\+out}. To find out how large to set the arrays, I executed \begin{DoxyVerb}grep positions ms.out | wc 
\end{DoxyVerb}


which produced the output \begin{DoxyVerb}1   89366 1161757
\end{DoxyVerb}


This indicates that I need to accomodate input lines with 89366 tokens and 1161757 characters. So I ran {\ttfamily ms2gtp} like this\+: \begin{DoxyVerb}ms2gtp --maxTokens 89370 --inBuffSize 1161800 ms.out > ms.gtp
\end{DoxyVerb}


This reads input from file {\ttfamily ms.\+out} and writes to file {\ttfamily ms.\+gtp}. The first few lines of {\ttfamily ms.\+gtp} look like this\+: \begin{DoxyVerb}# ms2gtp was run Sat Feb  2 11:15:05 2013
# cmd line     = ms2gtp -t 89370 -b 1161800 ms.out
# sim cmd      = ms 50 1 -t 20000 -r 20000 1000000
# nnucleotides = 1000000
# segsites     = 89365
# nsequences   = 50
# recomb rate per site = 1e-06
# ploidy       = 1
#   snp_id     nucpos         mappos alleles genotypes
         0          1 0.000000006802      01 0011000000...
         1          6 0.000000061255      01 0000000000...
\end{DoxyVerb}


In this example, I've truncated the genotypes on the right in order to fit them on the page.

Here is the usage message\+: \begin{DoxyVerb}usage: ms2gtp [options] [input_file_name]
   where options may include:
   -t \<x\> or --maxTokens \<x\>
      maximum tokens per input line
   -b \<x\> or --inBuffSize \<x\>
      size of input buffer
   -R \<x\> or --recombination \<x\>
      rate for adjacent nucleotides
   -h     or --help
      print this message
  To figure out appropriate values for -t and -b, execute the
  following at the command line:

     grep positions ms.out | wc

  where "ms.out" is the ms output file. This command prints 3 numbers.
  Number of tokens must be as large as 2nd; buffer size as large as the
  3rd. If no input file is given, ms2gtp reads stdin. ms2gtp
  always writes to standard output.
\end{DoxyVerb}


\begin{DoxyCopyright}{Copyright}
Copyright (c) 2014, Alan R. Rogers \href{mailto:rogers@anthro.utah.edu}{\tt rogers@anthro.\+utah.\+edu}. This file is released under the Internet Systems Consortium License, which can be found in file \char`\"{}\+L\+I\+C\+E\+N\+S\+E\char`\"{}. 
\end{DoxyCopyright}
