use simplex-\/simulated annealing to estimate population history

`sald`\-: use simplex-\/simulated annealing to estimate population history ======================================================================

\-To estimate the parameters describing population history, we need to find values that provide the best fit to data, which include both linkage disequilibrium (measured by $\sigma_d^2$) and the site frequency spectrum. \-This involves maximization on a complex surface with lots of local peaks. (\-To verify this for yourself, see \hyperlink{mcmcld}{mcmcld}.) \-To avoid getting stuck on local peaks, `sald` uses the simplex version of simulated annealing.

\-Usage -\/-\/-\/-\/-\/

\-The input data file should be as produced by \hyperlink{eld}{eld}.

\-Although simulated annealing works pretty well, \-I often find that different runs end up on different peaks. \-Therefore, `sald` is able to launch multiple simulated annealing jobs, each from a random starting point. \-These can run in parallel, on separate threads. \-The number of parallel optimizers is set using the `-\/-\/n\-Opt` argument described below.

\-The program also deals with bootstrap data, as provided by \hyperlink{eld}{eld}. \-The various boostrap data sets also run in parallel if your machine has multiple cores. \-By default `sald` does not process bootstrap replicates\-: use `-\/-\/bootfile` if you want it to.

\-By default `sald` uses as many threads as your machine has cores. \-This is not a good idea if you are sharing a machine with other users. \-Set the number of threads to some smaller number using `-\/-\/threads`. \-On linux or osx, you can use `top` to figure out how many threads are actually running. \-If you launch 20 threads but only 10 run at any given time, your job will run slower. \-Stop it and launch again with `-\/-\/threads 10`.

\-Simulated annealing works by beginning with a flattened version of the objective function. \-In this flattened version, all the peaks are smaller, so it is easy for the simplex to move from peak to peak. \-The extent of flattening is controlled by a parameter called \char`\"{}temperature\char`\"{}. \-High temperature implies lots of flattening. \-The annealing algorithm runs for awhile at a high temperature, then lowers the temperature and runs awhile more. \-The succession of temperatures and the number of iterations at each temperature is called the \char`\"{}annealing schedule\char`\"{}. \-You can change the performance of the algorithm by adjusting this schedule. \-See the `-\/-\/init\-Tmptr`, `-\/-\/n\-Per\-Tmptr` and `-\/-\/tmptr\-Decay` arguments.

\-If a bootstrap file is specified using the -\/f or -\/-\/bootfile option, `sald` will also write a file containing the parameter estimates for each bootstrap replicate. \-This file has a name like $<$input\-\_\-file\-\_\-name$>$-\/$<$jobid$>$.fboot, where $<$input\-\_\-file\-\_\-name$>$ is the base name of the input file, and $<$jobid$>$ is a hexadecimal number intended to uniquely identify the current job. \-This .fboot file can be used as input to the program `tabfboot`.

usage\-: sald \mbox{[}options\mbox{]} input\-\_\-file\-\_\-name where options may include\-: -\/m $<$method$>$ or -\/-\/methods $<$method$>$ specify method \char`\"{}\-Hill\char`\"{}, or \char`\"{}\-Strobeck\char`\"{}, or \char`\"{}\-Hill,\-Strobeck\char`\"{} -\/n $<$x$>$ or -\/-\/two\-Nsmp $<$x$>$ haploid sample size -\/t $<$x$>$ or -\/-\/threads $<$x$>$ number of threads (default is auto) -\/u $<$x$>$ or -\/-\/mutation $<$x$>$ mutation rate/generation -\/v $<$x$>$ or -\/-\/verbose more output -\/f $<$x$>$ or -\/-\/bootfile $<$x$>$ read bootstrap file x -\/c $<$x$>$ or -\/-\/confidence $<$x$>$ specify confidence level for \-C\-Is of parameters -\/-\/two\-N $<$x$>$ haploid pop size to x in current epoch -\/\-T $<$x$>$ or -\/-\/time $<$x$>$ length of current epoch (generations) -\/\-E or -\/-\/nextepoch move to next earlier epoch -\/-\/no\-Random\-Start \-Don't initialize \hyperlink{struct_pop_hist}{\-Pop\-Hist} at random -\/-\/n\-Opt $<$x$>$ optimizers per data set -\/-\/init\-Tmptr $<$x$>$ initial temperature -\/-\/n\-Tmptrs $<$x$>$ number of temperatures -\/-\/tmptr\-Decay $<$x$>$ ratio of successive temperatures -\/i $<$x$>$ or -\/-\/n\-Itr $<$x$>$ total number of iterations -\/h or -\/-\/help print this message

\begin{DoxyCopyright}{\-Copyright}
\-Copyright (c) 2014, 2015, \-Alan \-R. \-Rogers $<$\href{mailto:rogers@anthro.utah.edu}{\tt rogers@anthro.\-utah.\-edu}$>$. \-This file is released under the \-Internet \-Systems \-Consortium \-License, which can be found in file \char`\"{}\-L\-I\-C\-E\-N\-S\-E\char`\"{}. 
\end{DoxyCopyright}
